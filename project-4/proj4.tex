% @file proj4.tex
% @brief FIT VUT -- ITY -- project 4
% @author Matyas Strelec <xstrel03@stud.fit.vutbr.cz>
% @date 2023-04-15

\documentclass[a4paper, 11pt]{article}
\usepackage[left=2cm, top=3cm, text={17cm, 24cm}]{geometry}
\usepackage{color}
\usepackage[utf8]{inputenc}
\usepackage[czech]{babel}
\usepackage[colorlinks=true,urlcolor=black,citecolor=black]{hyperref}
\usepackage{graphicx}

\begin{document}

\begin{titlepage}
    \begin{center}
		{\Huge{\textsc{Vysoké učení technické v~Brně}}\\
        \huge{\textsc{Fakulta informačních technologií}}\\}
		\vspace{\stretch{0.382}}
		\LARGE{Typografie a~publikování\,{}--\,{}4.\ projekt}\\
        \Huge{Historie Typografie\\}
		\vspace{\stretch{0.618}}
        {\Large{\today \hfill Matyáš Strelec}}
	\end{center}
\end{titlepage}

\pagebreak

\section{Úvod}
Typografie má historii sahající do dávné minulosti, v tomto článku se budu věnovat
čtyřem nejdůležitějším obdobím vývoje typografie.

\section{Starověk}
Ve starověkém egyptě se hieroglyfy objevily přibližně v roce 3000 př.n.l., malované na podložku a poté
vyřezávané do kamene. Byly používány hlavně panovníky a pro náboženské účely. Později bylp vytvořeno nové
písmo, které se nazývalo hieratické. Používalo se zejména na oficiální dokumenty. Bylo zjednodušené a psáno 
inkoustem na papyrusu, pro což bylo vhodnější než původní hieroglyfy\cite{betro1996}\cite{Kleckova2015}.

\section{Knihtisk}
V roce 1440 vynalezl Johannes Gutenberg mechanický tiskový stroj, což vyústilo v revoluci v předávání 
informací\cite{barbier2017}. Gutenberg ale nebyl první, kdo využil techonologii podobnou knihtisku.
Podobné způsoby tisku byly používány v Číně v 11. století\cite{duchesne2006}\cite{musson1958}.


\section{Průmyslová revoluce}
Průmyslová revoluce změnila směřování tisku nejen mechanizováním ručního řemesla, ale i zvýšením poptávky po
z něj vycházejících výrobků. V 19. století se vynálezci potýkali s mnoha problémy v produkci papíru,
složení, tisku a vazbě. Řešením, které nejvíce ovlivnilo vzhled knihy, bylo strojové skládání knih.
Tyto stroje však vyžadovaly určitá omezení ve stylu písma a jeho složitosti a podobně\cite{history2018}.

\section{Digitalizace}
V roce 1968 vytvořil Rudolf Hell první digitální písmo nazývané Digi Grotesk. Digitální písmo však v 
začátcích nebylo ideální, neboť bylo vytvořeno bitmapově, čímž byla značně snížena jeho kvalita a s ní 
i čitelnost. O 6 let později bylo vytvořeno první vektorové digitální písmo, což umožnilo mnohem snazší práci
s písmem, jeho čitelnost, a snížilo požadavky na paměť. Díky textovým editorům se práce s písmem stala skoro
každodenní záležitostí pro většinu lidí\cite{bigelow1983}\cite{plna2006}.

\newpage
\bibliographystyle{czechiso}
\renewcommand{\refname}{Literatura}
\bibliography{proj4}

\end{document}

