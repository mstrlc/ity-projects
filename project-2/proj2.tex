% @file proj2.tex
% @brief FIT VUT -- ITY -- project 2
% @author Matyas Strelec <xstrel03@stud.fit.vutbr.cz>
% @date 2023-02-22

\documentclass[11pt, twocolumn, a4paper]{article}
\usepackage[left=1.4cm, top=2.3cm, text={18.2cm, 25.2cm}]{geometry}
\usepackage[utf8]{inputenc}
\usepackage[czech]{babel}
\usepackage[IL2]{fontenc}
\usepackage{hyperref}
\usepackage{amsthm, amssymb, amsmath}
\usepackage{times}

\begin{document}

\begin{titlepage}
	\begin{center}
		% Namisto odradkovani 0.4 em a 0.5 em jsem pouzil pro obe odradkovani 0.1em,
		% ktere se vzoru nejvice blizi.
		\textsc{
			\Huge{Vysoké učení technické v~Brně}\\[0.1em]
			\huge{Fakulta informačních technologií\\}
		}
		\vspace{\stretch{0.382}}
		\LARGE{
			Typografie a~publikování\,{}--\,{}2.\ projekt\\[0.1em]
			Sazba dokumentů a~matematických výrazů\\
		}
		\vspace{\stretch{0.618}}
	\end{center}
	{\Large{
		2023
		\hfill
		Matyáš Strelec (xstrel03)
		}
	}
\end{titlepage}

\newpage

\label{1}

\section*{Úvod}
V~této úloze si vyzkoušíme sazbu titulní strany, matematických vzorců, prostředí
a~dalších textových struktur obvyklých pro technicky zaměřené texty\,{}--\,{}například
Definice~\ref{thm_1} nebo rovnice~\eqref{eq_3} na straně~\pageref{1}. Pro vytvoření těchto
odkazů používáme kombinace příkazů \verb|\label|, \verb|\ref|, \verb|\eqref| a~\verb|\pageref|. Před odkazy
patří nezlomitelná mezera. Pro zvýrazňování textu jsou zde několikrát použity příkazy
\verb|\verb| a~\verb|\emph|.

Na titulní straně je použito prostředí \verb|titlepage| a~sázení nadpisu podle optického středu s~využitím \emph{přesného}
zlatého řezu. Tento postup byl probírán na přednášce. Dále jsou na titulní straně použity čtyři různé velikosti písma a~mezi
dvojicemi řádků textu je použito odřádkování se zadanou relativní velikostí 0,5\,em a~0,4\,em\footnote{Nezapomeňte použít správný typ mezery mezi číslem a~jednotkou.}.

\section{Matematický text}
V~této sekci se podíváme na sázení matematických symbolů a~výrazů v~plynulém textu pomocí prostředí \verb|math|. 
Definice a~věty sázíme pomocí příkazu \verb|\newtheorem| s~využitím balíku \verb|amsthm|. Někdy je vhodné použít konstrukci \verb|${}$|
nebo \verb|\mbox{}|, která říká, že (matematický) text nemá být zalomen.

\newtheorem{definition}{Definice}
\begin{definition}
	\label{def_1}
	\emph{Zásobníkový automat} (ZA) je definován jako sedmice tvaru $A = (Q, \Sigma, 
	\Gamma, \delta, q_0, Z_0, F)$, kde:
	\begin{itemize}
		\item $Q$ je konečná množina \emph{vnitřních (řídicích) stavů},
		\item $\Sigma$ je konečná \emph{vstupní abeceda},
		\item $\Gamma$ je konečná \emph{zásobníková abeceda},
		\item $\delta$ je \emph{přechodová funkce} $Q \times (\Sigma \cup \{
		      \epsilon \}) \times \Gamma \to 2^{Q \times \Gamma^*}$,
		\item $q_0 \in Q$ je \emph{počáteční stav}, $Z_0 \in \Gamma$ je 
		      \emph{startovací symbol zásobníku} a~$F \subseteq Q$ je množina \emph{koncových stavů}.
	\end{itemize}
\end{definition}
Nechť $P = (Q, \Sigma, \Gamma, \delta, q_0, Z_0, F)$ je ZA. \emph{Konfigurací} nazveme
trojici $(q, w, \alpha) \in Q \times \Sigma^* \times \Gamma^*$, kde $q$ je aktuální stav
vnitřního řízení, $w$ je dosud nezpracovaná část vstupního řetězce a~$\alpha = Z_{i_1} Z_{i_2} \dots Z_{i_k}$ je obsah zásobníku.

\subsection{Podsekce obsahující definici a~větu}
\begin{definition}
	\label{def_2}
	\emph{Řetězec $w$ nad abecedou $\Sigma$ je přijat ZA $A$}~jestliže $(q_0, w, Z_0) \overset{*}{\underset{A}{\vdash}} (q_F, \epsilon, \gamma)$
	pro nějaké $\gamma \in \Gamma^*$ a~$q_F \in F$. Množina $L(A) = \{w \mid w \text{ je přijat ZA }A\} \subseteq \Sigma^*$ je \emph{jazyk přijímaný ZA} $A$.
\end{definition}

\newtheorem{theorem}{Věta}
\begin{theorem}
	\label{thm_1}
	Třída jazyků, které jsou přijímány ZA, odpovídá \emph{bezkontextovým jazykům}.
\end{theorem}


\section{Rovnice}
Složitější matematické formulace sázíme mimo plynulý text pomocí prostředí \verb|displaymath|.
Lze umístit i~několik výrazů na jeden řádek, ale pak je třeba tyto vhodně oddělit,
například příkazem \verb|\quad|.
$$
1^{2^3} \neq \Delta^1_{\Delta^2_{\Delta^3}}
\quad
y^{11}_{22} - \sqrt[9]{x+\sqrt[7]{y}}
\quad
x > y_1 \leq y^2
$$
V~rovnici \eqref{eq_2} jsou využity tři typy závorek s~různou \emph{explicitně}
definovanou velikostí. Také nepřehlédněte, že nasledující tři rovnice mají zarovnaná
rovnítka, a~použijte k~tomuto účelu vhodné prostředí.
\begin{eqnarray}
	\label{eq_1} -\cos^2 \beta & = & \frac{\frac{\frac{1}{x}+\frac{1}{3}}{y}+1000}{\prod\limits _{j=2}^8 q_j} \\
	\label{eq_2} \bigg( \Big\{ b \star \big[ 3 \div 4 \big] \circ a \Big\}^\frac{2}{3}\bigg) & = & \log_{10}x \\
	\label{eq_3} \int_a^b f(x)\,\mathrm{d}x & = & \int_c^d f(y)\,\mathrm{d}y
\end{eqnarray}
V~této větě vidíme, jak vypadá implicitní vysázení limity $\lim_{m \to \infty} f(m)$ v~normálním
odstavci textu. Podobně je to i~s~dalšími symboly jako $\bigcup_{N \in \mathcal{M}} N$ či 
$\sum_{i=1}^{m} x_i^2$. S~vynucením méně úsporné sazby příkazem \verb|\limits|
budou vzorce vysázeny v~podobě $\lim\limits_{m \to \infty} f(m)$
a~$\sum\limits_{i=1}^{m} x_i^4$.

\section{Matice}
Pro sázení matic se velmi často používá prostředí \verb|array| a~závorky (\verb|\left|, \verb|\right|).
$$
\mathbf{B} =
\left|
\begin{array}{c c c c}
	b_{11} & b_{12} & \dots  & b_{1n} \\
	b_{21} & b_{22} & \dots  & b_{2n} \\
	\vdots & \vdots & \ddots & \vdots \\
	b_{m1} & b_{m2} & \dots  & b_{mn} 
\end{array}
\right|
=
\left|
\begin{array}{c c}
	t & u \\
	v & w 
\end{array}
\right|
= tw - uv
$$

$$
\mathbb{X}=\mathbf{Y}\Longleftrightarrow\bigg[
	\begin{array}{c c c}
		\phantom{} & \Omega+\Delta & \hat{\psi} \\
		\vec{\pi}  & \omega        & \phantom{} \\
	\end{array}
\bigg]
\neq 42
$$

Prostředí \verb|array| lze úspěšně využít i~jinde, například na pravé straně následující rovnice.
Kombinační číslo na levé straně vysázejte pomocí příkazu \verb|\binom|.

$$
\binom{n}{k} =
\left\{
\begin{array}{c l}
	0                   & \text{pro } k < 0           \\
	\frac{n!}{k!(n-k)!} & \text{pro } 0 \leq k \leq n \\
	0                   & \text{pro } k > 0           
\end{array}
\right.
$$

\end{document}